\chapter{Considerações finais}
\label{consideraçao}

Nesta tese, apresentou-se uma proposta de modelo que integra o \ac{PESD} com a
estimativa do valor de \ac{HC} de recursos energéticos distribuídos. Este modelo
foi desenvolvido a partir de uma formulação baseada no modelo proposto por
\citeonline{MunozDelgado2015}.

Esta proposta baseou-se nos conceitos de \ac{HC} encontrados na literatura para
introduzi-lo no contexto de \ac{PESD}, formulando um novo modelo com uma nova
função objetivo e restrições a fim de estimar o valor de \ac{HC}. Esta
formulação envolve um espelhamento das restrições do comportamento do sistema de
distribuição, permitindo simular a presença de novos geradores distribuídos. A
integração entre o modelo de \ac{PESD} e o modelo proposto de \ac{HC} é
realizada através de uma modelagem bi-objetiva, que é solucionada utilizando a
união dos métodos de $\epsilon$-restrito e otimização hierárquica.

A eficácia da proposta é demonstrada através dos resultados obtidos a partir da
aplicação do modelo proposto em três sistemas teste baseados nos encontrados na
literatura. Os resultados apontam que o modelo bi-objetivo proposto resulta em
várias opções de decisões de investimento que demonstram a concorrência entre os
objetivos de redução de custos com sistema planejado e aumento do valor de
\ac{HC}. Além disso, foram observados que mudanças topológicas podem causar
impactos no valor de \ac{HC}, e que a relação entre os custos com o \ac{PESD} e
valor de \ac{HC} devem ser investigados caso-a-caso.

A principal contribuição dos modelos apresentados é a integração dos modelos
clássicos de \ac{PESD} com a estimativa do valor de \ac{HC}. A partir destes
modelos, também foi possível observar quantativamente a relação entre os custos
com o \ac{PESD} e a melhoria do valor de \ac{HC}. Observou-se que, até o momento
da escrita desta tese, não foram encontrados trabalhos acadêmicos similares que
demonstraram essa relação nem uma forma de estima-la.

Entretanto, há diversas oportunidades de melhoria no modelo apresentado nesta
tese. Um exemplo de melhoria seria a adição de uma análise mais estatística do
comportamento de cargas, geração renovável e avaliação do \ac{HC}, bem como a
inserção da modelagem de sistemas de armazenamento de energia durante a expansão
do sistema de distribuição. Portanto, na forma apresentada nesta tese, os
modelos devem ser interpretados como uma avaliação inicial de possíveis
soluções. Recomenda-se então que estas oportunidades sejam investigadas em
trabalhos futuros.

A seguir são apresentadas as publicações realizadas desde o início do doutorado.


\section{Publicações realizadas}

\noindent FILHO, SERGIO A. DE MORAIS ; \textbf{MONTEIRO, FELIPE M. DOS S.} ; VIEIRA, JOSE CARLOS M. ; ASADA, EDUARDO N. ; BICZKOWSKI, MAURICIO . Impact Index for Allocating Transportable Energy Storage Systems in Power Distribution Networks. In: \textit{2019 IEEE Milan PowerTech}, IEEE, 2019. 

\noindent Disponível em:
https://ieeexplore.ieee.org/document/8810675/

\vspace{1cm}
\noindent DE SOUZA, JONAS V. ; MOMESSO, ANTONIO E. C. ; \textbf{MONTEIRO, FELIPE M. DOS S.} ; OTTO, RODRIGO B. ; ASADA, EDUARDO N. . Intelligent Management of Battery System for Energy Arbitrage. In: \textit{2019 IEEE Milan PowerTech}, IEEE, 2019. 

\noindent Disponível em:
https://ieeexplore.ieee.org/document/8810805

\vspace{1cm}
\noindent  DE SOUZA, JONAS V. ; FARIA, WANDRY R. ; \textbf{MONTEIRO, FELIPE M. DOS S.} ; OTTO, RODRIGO B. ; BICZKOWSKI, MAURICIO ; ASADA, EDUARDO N. . Battery Energy Storage System Allocation in Distribution Systems for Power Loss and Operational Costs Reduction. In: \textit{2019 IEEE PES Innovative Smart Grid Technologies Conference Latin America (ISGT Latin America)}, IEEE, 2019.

\noindent Disponível em:
https://ieeexplore.ieee.org/document/8810805

\newpage
\vspace{1cm}
\noindent \textbf{MONTEIRO, F. M. DOS S.}; SOUZA. J. V.; ASADA, E. Multistage Distribution System Expansion Planning with many Alternatives of Conductors. In: \textit{ 2020 IEEE Power \& Energy Society General Meeting (PESGM)}. IEEE, 2020. 

\noindent Disponível em: https://ieeexplore.ieee.org/document/9281945

\vspace{1cm}
\noindent SOUZA. J. V.; \textbf{MONTEIRO, F. M. DOS S.};  ASADA, E.; OTTO, R.; BICZKOWSKI, M. Management of an Electrical Storage System for Joint Energy Arbitrage and Improvement of Voltage Profile. In: \textit{5th Workshop on Communication Networks and Power Systems (WCNPS 2020)}. IEEE, 2020. 

\noindent Disponível em:  https://ieeexplore.ieee.org/document/9263720


\vspace{1cm}
\noindent KUME, G.; MOMESSO, A.; \textbf{MONTEIRO, F.}; ASADA, E. Impedance-based Fault Location Error Analysis in Distribution Network with Distributed Generation. In: \textit{Anais do VIII Simpósio Brasileiro de Sistemas Elétricos}. SBA, 2020.

\noindent Disponível em: https://doi.org/10.48011/sbse.v1i1.2395


\vspace{1cm}
\noindent SOUZA. J. V.; \textbf{MONTEIRO, F. M. S.};  GODOY, P. T.; ASADA, E. Active distribution network expansion planning considering microgrid operation and reliability. In: \textit{ 2022 IEEE Power \& Energy Society General Meeting (PESGM)}. IEEE, 2022.

\noindent Disponível em:  https://ieeexplore.ieee.org/document/9916750



\vspace{1cm}
\noindent SOUZA. E. C.; \textbf{MONTEIRO, F. M. S.}; ASADA, E. Challenges of Developing Situational Awareness in Active Distribution Systems. In: \textit{Anais do XXIV Congresso Brasileiro de Automática}. SBA, 2022.

\noindent Disponível em:  https://sba.org.br/cba2022/anais-do-cba-2022/



\vspace{1cm}
\noindent \textbf{MONTEIRO, F. M. S.}; SOUZA. J. V.; ASADA, E. Analytical method to estimate the steady-state voltage impact of Non-Utility Distributed Energy Resources. \textit{Electric Power Systems Research (218)}. Elsevier, 2023.

\noindent Disponível em: https://doi.org/10.1016/j.epsr.2023.109190






\iffalse
\vspace{-0.5cm}


\section{Publicações realizadas desde o início do doutorado}

\noindent FILHO, SERGIO A. DE MORAIS ; \textbf{MONTEIRO, FELIPE M. DOS S.} ; VIEIRA, JOSE CARLOS M. ; ASADA, EDUARDO N. ; BICZKOWSKI, MAURICIO . Impact Index for Allocating Transportable Energy Storage Systems in Power Distribution Networks. In: 2019 IEEE Milan PowerTech, 2019, Milan. 2019 IEEE Milan PowerTech, 2019. p. 1. 

\noindent Disponível em:
https://ieeexplore.ieee.org/document/8810675/

\vspace{1cm}
\noindent DE SOUZA, JONAS V. ; MOMESSO, ANTONIO E. C. ; \textbf{MONTEIRO, FELIPE M. DOS S.} ; OTTO, RODRIGO B. ; ASADA, EDUARDO N. . Intelligent Management of Battery System for Energy Arbitrage. In: 2019 IEEE Milan PowerTech, 2019, Milan. 2019 IEEE Milan PowerTech, 2019. p. 1. 

\noindent Disponível em:
https://ieeexplore.ieee.org/document/8810805

\vspace{1cm}
\noindent  DE SOUZA, JONAS V. ; FARIA, WANDRY R. ; \textbf{MONTEIRO, FELIPE M. DOS S.} ; OTTO, RODRIGO B. ; BICZKOWSKI, MAURICIO ; ASADA, EDUARDO N. . Battery Energy Storage System Allocation in Distribution Systems for Power Loss and Operational Costs Reduction. In: 2019 IEEE PES Innovative Smart Grid Technologies Conference Latin America (ISGT Latin America), 2019, Gramado. 2019 IEEE PES Innovative Smart Grid Technologies Conference - Latin America (ISGT Latin America), 2019. p. 1. 

\noindent Disponível em:
https://ieeexplore.ieee.org/document/8810805

\vspace{1cm}
\noindent \textbf{MONTEIRO, F. M. DOS S.}; SOUZA. J. V.; ASADA, E. Multistage Distribution System Expansion Planning with many Alternatives of Conductors. In: \textit{ 2020 IEEE Power \& Energy Society General Meeting (PESGM)}. IEEE, 2020. 

\noindent Disponível em: https://ieeexplore.ieee.org/document/9281945

\vspace{1cm}
\noindent SOUZA. J. V.; \textbf{MONTEIRO, F. M. DOS S.};  ASADA, E.; OTTO, R.; BICZKOWSKI, M. Management of an Electrical Storage System for Joint Energy Arbitrage and Improvement of Voltage Profile. In: \textit{5th Workshop on Communication Networks and Power Systems (WCNPS 2020)}. IEEE, 2020. 

\noindent Disponível em:  https://ieeexplore.ieee.org/document/9263720
\newpage

\vspace{1cm}
\noindent KUME, G.; MOMESSO, A.; \textbf{MONTEIRO, F.}; ASADA, E. Impedance-based Fault Location Error Analysis in Distribution Network with Distributed Generation. In: \textit{Anais do VIII Simpósio Brasileiro de Sistemas Elétricos}. SBA, 2020.

\noindent Disponível em: https://sbse2020.galoa.com.br/


 \section{Publicações em desenvolvimento}
 
 \noindent \textbf{MONTEIRO, F. M. DOS S.}; SOUZA. J. V.; ASADA, E. Estimating the Impact of Non-Utility Distributed Energy Resources on Steady-state Voltage in Unmonitored Distribution Systems. \textit{IEEE Transactions on Sustainable Energy}. IEEE, 2021.

Este artigo apresenta uma proposta analítica da estimativa dos impactos na tensão que recursos energéticos distribuídos (incluindo geradores distribuídos). Este trabalho é resultados das simulações realizadas para entender como o sistema de distribuição se comporta com injeções de potência destes recursos. O artigo já foi desenvolvido e está em processo de correção.

\vspace{1cm}
\noindent \textbf{Hosting Capacity in Distribution Systems: Trends, Challenges, and Opportunities}
 
Este artigo será desenvolvido descrevendo as tendências observadas no levantamento bibliográfico deste trabalho. Espera-se publica-lo em um periódico acadêmico antes do término do doutorado.
 
\vspace{1cm}
\noindent \textbf{Estimating Hosting Capacity in Distribution Expansion Planning}

Este artigo será desenvolvido descrevendo o modelo proposto neste trabalho, após as correções e sugestões realizadas pela banca de avaliação desta qualificação. Espera-se publica-lo em um periódico acadêmico antes do término do doutorado.



\fi